% File:  statement.tex
% Author: mikolas
% Created on: 13 Feb 2018 13:38:26
% Copyright (C) 2018, Mikolas Janota
\documentclass[a4paper,12pt]{article}
%\usepackage[hmargin=2.5cm,vmargin=1.5cm,a4paper]{geometry}
\usepackage{xspace}
\usepackage{amsmath,amssymb}
\usepackage{paralist}
\usepackage{fancyhdr}
\usepackage{setspace}
\usepackage{tikz}
\usepackage{palatino} % Uncomment to use the Palatino font
\def\miko{Mikol\'a\v{s} Janota}
\newcommand{\firstPlace}{$\mathbf{1}^\textbf{st}$ \textbf{place}\xspace}
% ------------------------------- Hyperref -------------------------------------
\usepackage{color}
\definecolor{citeblue}{rgb}{0.1,0,.4}
\definecolor{refcolor}{rgb}{0,0,0.4}
\definecolor{midgreen}{RGB}{0,150,0}
\definecolor{darkgreen}{RGB}{0,128,0}
\definecolor{darkblue}{RGB}{0,0,128}
\definecolor{darkred}{RGB}{192,0,0}
\definecolor{dgray}{rgb}{.3,.3,.3}
\usepackage[pdftex%
,colorlinks=true%
,bookmarks=true%
,linkcolor=citeblue%
,citecolor=citeblue%
,urlcolor=blue%
,plainpages=false]{hyperref}
\hypersetup{pdfauthor={{\miko}}}
% ------------------------------- hdr -------------------------------------
\usepackage{fancyhdr}
\pagestyle{fancy}
%\headheight 16pt
\definecolor{dgray}{rgb}{.3,.3,.3}
%\rhead{\textcolor{dgray}{\PROJECT}}
%\chead{\textcolor{dgray}{Part B1}}
%\lhead{\textcolor{dgray}{Janota}}
%\cfoot{Page \thepage \ of 13}
%\cfoot{}
%\lfoot{\textcolor{dgray}{{\miko}, 2018}}
%\renewcommand{\footrulewidth}{0.4pt}
%\renewcommand{\headrulewidth}{0pt}
% -------------------------------------------------------------------------
\def\theTitle{CQESTO, QBF Solver}
\title{\theTitle}
\author{{\miko}\\
IST/INESC-ID,\\
University of Lisbon, Portugal\\
\href{mailto:mikolas.janota@gmail.com}{mikolas.janota@gmail.com}
}
% -------------------------------------------------------------------------
\begin{document}
%\setcounter{page}{1}
%{\Large\bf Instituto Superior T\'{e}cnico}
%\par\noindent\rule{\textwidth}{0.4pt}
%{\large\bf QFUN, QBF Solver}
%\noindent\rule{\textwidth}{0.4pt}
%\end{center}
\maketitle

CQESTO is a solver for \emph{quantified Boolean formulas (QBF)} that 
is inspired by my older solver QESTO~\cite{janota-ijcai15}.
The details of CQESTO can be found in my SAT 2018 publication~\cite{janota-sat18}.

Like in QESTO, CQESTO maintains a propositional formula $\alpha_i$
for each quantification level $i$.
This formula determines the possible moves of the corresponding player at that position.
Consequently, the formula is solved if one of $\alpha_1,\alpha_2$ becomes unsatisfiable.
The solver applies a SAT solver to solve each $\alpha_i$,
i.e.\ there may be exponentially many SAT calls throughout the run of the algorithm.
In this sense it is similar to other solvers like RAReQS~\cite{janota-sat11,janota-sat12,janota-ai16}. However, it is much more lightweight. CQESTO differs from QESTO 
by enabling calculation directly on circuits rather than CNF.
%
CQESTO is closely related to CAQE~\cite{RabeFMCAD15} and QuAbS~\cite{TentrupSAT16} as discussed in the CQESTO paper.

Compared to the 2017 submission, the solver is now  \emph{not} running the preprocessing script by Will Klieber,
which was struggling on some larger instances so in the end it didn't really pay off.
The rest is the same as 2017.

\noindent{\it Information about the solver:}

\noindent\begin{tabular}{ll}
     authors:& \miko\\
     source:& \url{https://github.com/MikolasJanota/cqesto}\\
     license:& GNU General Public License
\end{tabular}

% --------------------------------------------------------------------
\bibliographystyle{alpha}
\bibliography{refs}
% --------------------------------------------------------------------
\end{document}
